\documentclass[11pt, oneside]{article}   	% use "amsart" instead of "article" for AMSLaTeX format
\usepackage{geometry}                		% See geometry.pdf to learn the layout options. There are lots.
\geometry{letterpaper}                   		% ... or a4paper or a5paper or ... 
%\geometry{landscape}                		% Activate for for rotated page geometry
%\usepackage[parfill]{parskip}    		% Activate to begin paragraphs with an empty line rather than an indent
\usepackage{graphicx}				% Use pdf, png, jpg, or eps� with pdflatex; use eps in DVI mode
								% TeX will automatically convert eps --> pdf in pdflatex		
\usepackage{amssymb}

\title{A Microfluidic Device for Measuring Aerotaxis}
\author{Kiarash Adl, Brian Djaja, Katarina Struckmann and Logan Williams}
%\date{}							% Activate to display a given date or no date

\begin{document}
\maketitle
%\section{}
%\subsection{}
\section{Abstract}

Bacteria rely heavily on sensing environmental gradients to survive and reproduce. While significant effort has been made to understand responses to chemical gradients, or chemotactic responses, less literature exists characterizing responses to gas gradients, or aerotactic responses. This study employed a three-channel microfluidic device fabricated from PDMS to characterize aerotactic bacterial motion. In the presence of a gradient of nitrogen and oxygen, movement of the organism \textit{B. subtilis} tended toward oxygen and away from nitrogen, demonstrating an aerotactic response.

\section{Introduction}

The movement of bacteria in response to a gas gradient, aerotaxis, is a field that has emerged within the past 10 years. As such, it has not been studied to the extent of the similar process chemotaxis. Chemotaxis is the movement of bacteria in response to a chemical gradient, where the chemical can be anything from food to toxic molecules. Models for chemotaxis in bacteria have been characterized, both for populations and single bacterial cells.2,3

Aerotaxis was studied in a microfluidic device using a custom optical microscope. The microfluidic device allowed for development of a gas gradient along its length; the bacteria�s response to the gradient was visualized with the microscope. This setup can be utilized for studying different bacteria species and potentially other cell types. 

Aerotaxis has a broad range of applications; the signaling proteins utilized in for aerotactic responses in bacteria can be potentially engineered for turning on gene expression in the presence of a certain oxygen concentration, or lack thereof. Since aerotaxis occurs on the order of minutes, a response can be expressed fairly quickly. Additionally, modeling aerotaxis is key to understanding how applications such as the aforementioned can be utilized.

\section{Experimental apparatus}
\subsection{The microscope} % Logan or Kiarash?
\subsection{The microfluidic device} % Logan or Kiarash?
\subsection{Gas flow} % Kat?
\subsection{Bacteria culturing} % Brian?
\section{Results} % all of us
\section{Discussion} % all of us


\end{document}  